% ----------------------------------------------------------
\chapter{UC11 -  Definir Mês de Referência para Transações}
% ----------------------------------------------------------
\label{apendiceUC11}

\subsection*{Identificador}
\textit{UC11}

\subsection*{Título}
Definir Mês de Referência para Transações

\subsection*{Atores}
\begin{addmargin}[1.5cm]{0cm}
	\begin{itemize}
		\item Usuário
		\item Sistema de Gestão Financeira
	\end{itemize}
\end{addmargin}

\subsection*{Pré-condições}
\begin{addmargin}[1.5cm]{0cm}
	\begin{itemize}
		\item O usuário está autenticado no sistema.
		\item O usuário está registrando ou editando uma transação.
	\end{itemize}
\end{addmargin}

\subsection*{Fluxo Principal}
\begin{addmargin}[1.5cm]{0cm}
	\begin{enumerate}
		\item Durante o registro ou edição de uma transação, o usuário identifica a necessidade de definir um mês de referência.
		\item O sistema apresenta a opção para selecionar o mês de referência.
		\item O usuário seleciona o mês e ano desejados.
		\item O sistema valida a escolha.
		\item O mês de referência é associado à transação.
	\end{enumerate}
\end{addmargin}

\subsection*{Fluxos Alternativos}
\textit{N/A}

\subsection*{Fluxos de Exceção}
\textit{N/A}

\subsection*{Pós-condições}
\begin{addmargin}[1.5cm]{0cm}
	\begin{itemize}
		\item O mês de referência é definido para a transação, permitindo um controle orçamentário mais preciso.
	\end{itemize}
\end{addmargin}

\subsection*{Regras de Negócio}
\begin{addmargin}[1.5cm]{0cm}
	\begin{itemize}
		\item RN23: Todas as transações devem ter um mês de referência para controle orçamentário.
	\end{itemize}
\end{addmargin}

\subsection*{Notas e Comentários}
\begin{addmargin}[1.5cm]{0cm}
	\begin{itemize}
		\item A definição do mês de referência é especialmente crucial para transações de cartão de crédito, onde o impacto orçamentário pode não coincidir com a data da transação.
	\end{itemize}
\end{addmargin}