% ----------------------------------------------------------
\chapter{ADR-008 - Adoção da Arquitetura REST para Comunicação entre Front-end e Back-end Flask}
% ----------------------------------------------------------
\label{apendiceADR008}
% Conteúdo do Apêndice ADR-008

\subsubsection*{Title}
Adoção da Arquitetura REST para Comunicação entre Front-end e Back-end Flask.

\subsubsection*{Status}
Accepted

\subsubsection*{Context}
O \textit{Fintrackr} possui componentes de front-end e um back-end construído com Flask. Para garantir uma comunicação eficiente e padronizada entre esses dois componentes, é necessário escolher uma arquitetura de comunicação apropriada.

\subsubsection*{Decision}
Adotar a arquitetura REST (Representational State Transfer) como a principal forma de comunicação entre o front-end e o back-end Flask do Fintrackr.

\subsubsection*{Consequences}
\begin{itemize}
	\item \textbf{Padronização}: A arquitetura REST fornece um conjunto padronizado de convenções para a criação de APIs.
	\item \textbf{Escalabilidade}: REST é stateless, o que facilita a escalabilidade horizontal do sistema.
	\item \textbf{Flexibilidade}: Permite que o front-end e o back-end Flask se comuniquem de maneira eficiente.
	\item \textbf{Interoperabilidade}: Facilita a integração com sistemas externos ou de terceiros.
	\item \textbf{Simplicidade e Eficiência}: REST utiliza os métodos HTTP padrão e é baseado em recursos, tornando-o intuitivo e eficiente.
	\item \textbf{Desacoplamento}: Permite que o front-end e o back-end Flask evoluam separadamente, desde que a API permaneça consistente.
	\item \textbf{Conformidade com Padrões da Web}: REST é amplamente adotado na web, e muitas ferramentas e bibliotecas suportam essa arquitetura.
	\item \textbf{Considerações de Segurança}: Como qualquer API exposta, medidas de segurança adequadas, como autenticação e autorização, devem ser implementadas ao comunicar entre o front-end e o back-end Flask.
\end{itemize}

\newpage