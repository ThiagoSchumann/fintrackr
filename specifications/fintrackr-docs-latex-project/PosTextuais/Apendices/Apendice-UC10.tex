% ----------------------------------------------------------
\chapter{UC10 - Gerenciar Bancos}
% ----------------------------------------------------------
\label{apendiceUC10}

\subsection*{Identificador}
\textit{UC10}

\subsection*{Título}
Gerenciar Bancos

\subsection*{Atores}
\begin{addmargin}[1.5cm]{0cm}
    \begin{itemize}
        \item Usuário
        \item Sistema de Gerenciamento de Bancos
    \end{itemize}
\end{addmargin}

\subsection*{Pré-condições}
\begin{addmargin}[1.5cm]{0cm}
    \begin{itemize}
        \item O usuário está autenticado no sistema.
        \item O usuário possui acesso à seção de gerenciamento de bancos.
    \end{itemize}
\end{addmargin}

\subsection*{Fluxo Principal}
\begin{addmargin}[1.5cm]{0cm}
    \begin{enumerate}
        \item O usuário acessa a seção de gerenciamento de bancos.
        \item O sistema exibe a lista de bancos cadastrados pelo usuário.
        \item O usuário pode escolher adicionar, editar ou excluir informações de um banco.
        \item O sistema realiza a ação solicitada e atualiza a lista de bancos.
    \end{enumerate}
\end{addmargin}

\subsection*{Fluxos Alternativos}
\begin{addmargin}[1.5cm]{0cm}
    \begin{itemize}
        \item \textbf{Adicionar Novo Banco}:
        \begin{enumerate}
            \item O usuário seleciona a opção para adicionar um novo banco.
            \item O sistema apresenta um formulário para inserção de detalhes do banco.
            \item O usuário preenche os detalhes e submete o formulário.
            \item O sistema valida as informações e adiciona o novo banco à lista.
        \end{enumerate}
    
        \item \textbf{Editar Informações de um Banco Existente}:
        \begin{enumerate}
            \item O usuário seleciona um banco da lista e escolhe a opção para editar.
            \item O sistema apresenta um formulário preenchido com as informações atuais do banco.
            \item O usuário realiza as alterações desejadas e submete o formulário.
            \item O sistema valida e atualiza as informações do banco.
        \end{enumerate}
    \end{itemize}
\end{addmargin}

\subsection*{Fluxos de Exceção}
\begin{addmargin}[1.5cm]{0cm}
    \begin{itemize}
        \item \textbf{Informações do Banco Duplicadas}:
        \begin{enumerate}
            \item O sistema detecta que as informações inseridas para um novo banco ou após a edição já existem para outro banco registrado.
            \item O sistema informa ao usuário sobre a duplicidade e solicita correção.
        \end{enumerate}
    \end{itemize}
\end{addmargin}

\subsection*{Pós-condições}
\begin{addmargin}[1.5cm]{0cm}
    \begin{itemize}
        \item A lista de bancos do usuário é atualizada com as informações corretas.
    \end{itemize}
\end{addmargin}

\subsection*{Regras de Negócio}
\begin{addmargin}[1.5cm]{0cm}
    \begin{itemize}
        \item RN10: O email fornecido pelo usuário durante o registro ou atualização do perfil deve ser único no sistema.
        \item RN15: Informações de cartões ou contas duplicadas não são permitidas.
    \end{itemize}
\end{addmargin}

\subsection*{Notas e Comentários}
\begin{addmargin}[1.5cm]{0cm}
    \begin{itemize}
        \item O gerenciamento eficaz de informações bancárias é crucial para a integridade dos dados e a experiência do usuário.
    \end{itemize}
\end{addmargin}
