% ----------------------------------------------------------
\chapter{ADR-009 - Adoção do RabbitMQ para Comunicação entre o Microserviço da Planilha de Fatura e o Back-end Flask}
% ----------------------------------------------------------
\label{apendiceADR009}
% Conteúdo do Apêndice ADR-009

\subsubsection*{Title}
Adoção do RabbitMQ para Comunicação entre o Microserviço da Planilha de Fatura e o Back-end Flask.

\subsubsection*{Status}
Accepted

\subsubsection*{Context}
Para integrar o microserviço especializado da planilha de fatura com o back-end Flask do \textit{Fintrackr}, é crucial ter uma solução de mensageiria confiável que facilite a comunicação assíncrona, garanta a entrega de mensagens e suporte a escalabilidade.

\subsubsection*{Decision}
Adotar o RabbitMQ como a principal solução de mensageiria para garantir a comunicação eficiente entre o microserviço da planilha de fatura e o back-end Flask.

\subsubsection*{Consequences}
\begin{itemize}
	\item \textbf{Comunicação Assíncrona}: RabbitMQ permite que o microserviço e o back-end Flask se comuniquem de forma assíncrona, o que pode melhorar a eficiência e a resposta do sistema.
	\item \textbf{Garantia de Entrega}: As mensagens podem ser armazenadas e reenviadas em caso de falhas, garantindo a entrega.
	\item \textbf{Escalabilidade}: RabbitMQ suporta balanceamento de carga entre múltiplas instâncias, o que pode ajudar na escalabilidade da solução.
	\item \textbf{Flexibilidade}: Permite a integração com outros sistemas e serviços no futuro, graças ao seu modelo de mensageiria baseado em tópicos e filas.
	\item \textbf{Complexidade Adicional}: A introdução de uma solução de mensageira pode aumentar a complexidade do sistema e requerer monitoramento e gerenciamento adicionais.
	\item \textbf{Dependência Externa}: A adoção do RabbitMQ introduz uma dependência externa que precisa ser gerenciada e mantida.
\end{itemize}
