
% ----------------------------------------------------------
\chapter{ADR-005 - Front-end React - Component-Based Architecture}
% ----------------------------------------------------------
\label{apendiceADR005}
% Conteúdo do Apêndice ADR-005

\subsubsection*{Title}
Front-end React - Component-Based Architecture.

\subsubsection*{Status}
Accepted

\subsubsection*{Context}
O \textit{Fintrackr} visa fornecer uma interface de usuário intuitiva e eficiente para gerenciamento de finanças. A escolha da arquitetura do front-end é crucial para garantir uma experiência de usuário consistente e de alta qualidade.

\subsubsection*{Decision}
Adotar a arquitetura baseada em componentes usando o framework React para o front-end do \textit{Fintrackr}.

\subsubsection*{Consequences}
\begin{itemize}
	\item \textbf{Modularidade}: A arquitetura baseada em componentes permite modularidade, facilitando o desenvolvimento, teste e manutenção de partes individuais da interface do usuário.
	\item \textbf{Reusabilidade}: Componentes podem ser reutilizados em diferentes partes da aplicação ou até mesmo em outros projetos.
	\item \textbf{Eficiência}: Com o Virtual DOM do React, as atualizações da interface do usuário são otimizadas para serem rápidas e eficientes.
	\item \textbf{Comunidade}: React tem uma grande comunidade e ecossistema, oferecendo muitas bibliotecas e ferramentas auxiliares.
	\item \textbf{Curva de Aprendizado}: Pode haver uma curva de aprendizado para desenvolvedores não familiarizados com React ou arquitetura baseada em componentes.
	\item \textbf{Dependências}: Dependência de bibliotecas e ferramentas específicas do ecossistema React.
\end{itemize}
