
% ----------------------------------------------------------
\chapter{ADR-006 - Back-end Flask - Camadas com Clean/Onion}
% ----------------------------------------------------------
\label{apendiceADR006}
% Conteúdo do Apêndice ADR-006

\subsubsection*{Title}
Back-end Flask - Camadas com Clean/Onion.

\subsubsection*{Status}
Accepted

\subsubsection*{Context}
O \textit{Fintrackr} tem a responsabilidade de gerenciar a lógica de negócios relacionada ao gerenciamento de finanças. A escolha da arquitetura do back-end é essencial para garantir que a aplicação seja flexível, testável e manutenível.

\subsubsection*{Decision}
Adotar uma combinação de arquitetura em camadas com Clean/Onion para o back-end do \textit{Fintrackr} construído com Flask.

\subsubsection*{Consequences}
\begin{itemize}
	\item \textbf{Separação de Responsabilidades}: Esta abordagem garante que as responsabilidades sejam claramente separadas em diferentes camadas.
	\item \textbf{Flexibilidade}: Facilita mudanças e evolução do código ao manter a lógica de negócios separada da infraestrutura.
	\item \textbf{Testabilidade}: Torna mais fácil escrever testes unitários para a lógica de negócios.
	\item \textbf{Manutenibilidade}: O código torna-se mais fácil de manter e entender.
	\item \textbf{Curva de Aprendizado}: Pode haver uma curva de aprendizado para desenvolvedores não familiarizados com a arquitetura Clean/Onion.
	\item \textbf{Complexidade}: A introdução de várias camadas e a separação rigorosa podem aumentar a complexidade inicial.
\end{itemize}

\newpage
