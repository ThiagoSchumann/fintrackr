% ----------------------------------------------------------
\chapter{ADR-010 - Adoção do SQLAlchemy para Comunicação com o Banco de Dados no Flask}
% ----------------------------------------------------------
\label{apendiceADR010}
% Conteúdo do Apêndice ADR-010

\subsubsection*{Title}
Adoção do SQLAlchemy como ORM para Comunicação com o Banco de Dados no Flask.

\subsubsection*{Status}
Aceito

\subsubsection*{Context}
O \textit{Fintrackr}, construído usando Flask, precisa interagir de maneira eficiente e segura com o banco de dados. A escolha de uma estratégia ou ferramenta adequada para esta comunicação é crucial para a robustez e manutenibilidade do sistema.

\subsubsection*{Decision}
Adotar o SQLAlchemy, um ORM popular e amplamente utilizado na comunidade Flask, como a principal ferramenta para estabelecer a comunicação entre a aplicação Flask e o banco de dados.

\subsubsection*{Consequences}
\begin{itemize}
	\item \textbf{Abstração}: O SQLAlchemy permite que a equipe de desenvolvimento trabalhe com o banco de dados usando a linguagem Python, abstraindo muitos detalhes do SQL.
	\item \textbf{Segurança}: O SQLAlchemy tem proteções embutidas contra injeções SQL, reduzindo o risco desses tipos de ataques.
	\item \textbf{Produtividade}: Facilita operações comuns do banco de dados, como inserções, atualizações, leituras e exclusões, através de uma API intuitiva.
	\item \textbf{Portabilidade}: O SQLAlchemy suporta vários sistemas de gerenciamento de banco de dados, tornando mais fácil mudar para um banco de dados diferente no futuro, se necessário.
	\item \textbf{Complexidade Adicional}: Apesar de ser poderoso, pode haver uma curva de aprendizado inicial para desenvolvedores não familiarizados com o SQLAlchemy ou seus padrões.
\end{itemize}

\newpage
