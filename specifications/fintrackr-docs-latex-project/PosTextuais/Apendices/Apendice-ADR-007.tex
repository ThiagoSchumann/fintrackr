
% ----------------------------------------------------------
\chapter{ADR-007 - Microserviço com Pandas - Hexagonal}
% ----------------------------------------------------------
\label{apendiceADR007}
% Conteúdo do Apêndice ADR-007

\subsubsection*{Title}
Microserviço com Pandas - Hexagonal.

\subsubsection*{Status}
Accepted

\subsubsection*{Context}
O \textit{Fintrackr} tem uma necessidade específica de processar dados de planilhas para integração com o sistema. Esta funcionalidade é distinta das funções principais da aplicação e pode necessitar de escalabilidade e adaptabilidade independentes.

\subsubsection*{Decision}
Adotar a arquitetura hexagonal para o microserviço especializado em processamento de dados com Pandas.

\subsubsection*{Consequences}
\begin{itemize}
	\item \textbf{Flexibilidade}: A arquitetura hexagonal permite que o microserviço seja facilmente adaptado para diferentes interfaces ou integrações no futuro.
	\item \textbf{Testabilidade}: Facilita a escrita de testes, pois a lógica de negócios é desacoplada dos adaptadores.
	\item \textbf{Isolamento}: Falhas ou problemas no microserviço não afetarão outras partes da aplicação.
	\item \textbf{Especialização}: O microserviço pode ser otimizado para processar planilhas de forma eficiente.
	\item \textbf{Curva de Aprendizado}: Pode haver uma curva de aprendizado para desenvolvedores não familiarizados com a arquitetura hexagonal.
	\item \textbf{Gerenciamento Adicional}: Requer monitoramento, logging e rastreamento adicionais para gerenciar o microserviço.
\end{itemize}
