
% ----------------------------------------------------------
\chapter{ADR-002 - Adoção de Arquitetura Monolítica para a Lógica de Negócios Principal do \textit{Fintrackr}}
% ----------------------------------------------------------
\label{apendiceADR002}
% Conteúdo do Apêndice ADR-002

\subsubsection*{Title}
Adoção de Arquitetura Monolítica para a Lógica de Negócios Principal do \textit{Fintrackr}.

\subsubsection*{Status}
Accepted

\subsubsection*{Context}
O projeto \textit{Fintrackr} visa fornecer uma solução robusta para o gerenciamento de finanças. Uma parte significativa do sistema é composta por operações de cadastro e lógica de negócios. Para acomodar essas necessidades de maneira coesa, é essencial considerar uma abordagem arquitetural que possa integrar eficientemente esses componentes.

\subsubsection*{Decision}
Adotar uma arquitetura monolítica para a lógica de negócios principal e cadastros do \textit{Fintrackr}.

\subsubsection*{Consequences}
\begin{itemize}
	\item \textbf{Consistência}: Um monólito oferece uma base de código unificada, o que pode simplificar o desenvolvimento e a manutenção.
	\item \textbf{Performance}: Com todos os componentes principais no mesmo processo, as latências de comunicação são minimizadas.
	\item \textbf{Simplicidade}: Menos preocupações com a comunicação entre serviços.
	\item \textbf{Escalabilidade Vertical}: O monólito pode ser escalado verticalmente, mas pode haver limitações na escalabilidade horizontal.
	\item \textbf{Potencial de Acoplamento}: Pode haver riscos de acoplamento apertado entre os módulos ao longo do tempo.
	\item \textbf{Curva de aprendizado}: Desenvolvedores familiarizados com microserviços podem precisar de tempo para se adaptar à abordagem monolítica.
	\item \textbf{Flexibilidade}: Modificações no sistema podem requerer reimplantações completas em vez de serviços individuais.
	\item \textbf{Manutenção}: À medida que o projeto cresce, a manutenção pode se tornar mais desafiadora devido ao tamanho do código base.
\end{itemize}
