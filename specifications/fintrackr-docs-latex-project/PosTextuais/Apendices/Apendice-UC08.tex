% ----------------------------------------------------------
\chapter{UC08 - Gerenciar Cartões e Faturas}
% ----------------------------------------------------------
\label{apendiceUC08}

\subsection*{Identificador}
\textit{UC08}

\subsection*{Título}
Gerenciar Cartões e Faturas

\subsection*{Atores}
\begin{addmargin}[1.5cm]{0cm}
    \begin{itemize}
        \item Usuário
        \item Sistema de Gerenciamento Financeiro
    \end{itemize}
\end{addmargin}

\subsection*{Pré-condições}
\begin{addmargin}[1.5cm]{0cm}
    \begin{itemize}
        \item O usuário está autenticado e possui acesso à seção de gerenciamento de cartões.
    \end{itemize}
\end{addmargin}

\subsection*{Fluxo Principal}
\begin{addmargin}[1.5cm]{0cm}
    \begin{enumerate}
        \item O usuário acessa a seção de gerenciamento de cartões.
        \item O usuário visualiza seus cartões cadastrados.
        \item O usuário seleciona um cartão específico para visualizar detalhes e faturas associadas.
        \item O usuário pode adicionar, editar ou excluir informações do cartão.
        \item O usuário pode visualizar, adicionar, editar ou excluir faturas associadas ao cartão.
    \end{enumerate}
\end{addmargin}

\subsection*{Fluxos Alternativos}
\begin{addmargin}[1.5cm]{0cm}
    \begin{itemize}
        \item \textbf{Adicionar Novo Cartão}:
        \begin{enumerate}
            \item O usuário seleciona a opção para adicionar um novo cartão.
            \item O sistema apresenta um formulário para inserção de detalhes do cartão.
            \item O usuário preenche os detalhes e submete o formulário.
            \item O sistema valida as informações e adiciona o novo cartão à lista do usuário.
        \end{enumerate}
        
        \item \textbf{Excluir Cartão}:
        \begin{enumerate}
            \item O usuário seleciona um cartão e escolhe a opção para excluir.
            \item O sistema solicita confirmação.
            \item O usuário confirma a exclusão.
            \item O sistema remove o cartão e todas as faturas associadas.
        \end{enumerate}
    \end{itemize}
\end{addmargin}

\subsection*{Fluxos de Exceção}
\begin{addmargin}[1.5cm]{0cm}
    \begin{itemize}
        \item \textbf{Erro ao Adicionar Cartão}:
        \begin{enumerate}
            \item O sistema detecta um erro durante a adição do cartão (por exemplo, informações inválidas).
            \item O sistema informa ao usuário sobre o erro.
        \end{enumerate}
        
        \item \textbf{Erro ao Excluir Cartão}:
        \begin{enumerate}
            \item O sistema detecta um erro durante a exclusão do cartão (por exemplo, faturas pendentes).
            \item O sistema informa ao usuário sobre o erro.
        \end{enumerate}
    \end{itemize}
\end{addmargin}

\subsection*{Pós-condições}
\begin{addmargin}[1.5cm]{0cm}
    \begin{itemize}
        \item As informações do cartão e faturas associadas são atualizadas conforme as ações do usuário.
    \end{itemize}
\end{addmargin}

\subsection*{Requisitos}
\begin{addmargin}[1.5cm]{0cm}
	\begin{itemize}
        \item \textbf{RF08}: Facilitar o gerenciamento de cartões de crédito, incluindo registro de faturas e lançamentos associados.
	\end{itemize}
\end{addmargin}

\subsection*{Regras de Negócio}
\begin{addmargin}[1.5cm]{0cm}
    \begin{itemize}
        \item \textbf{RN02}: Cartões e contas com faturas ou transações associadas não podem ser excluídos.
        \item \textbf{RN07}: Despesas em cartões de crédito devem ser associadas à fatura corrente.
        \item \textbf{RN15}: Informações de cartões ou contas duplicadas não são permitidas.
    \end{itemize}
\end{addmargin}

\subsection*{Notas e Comentários}
\begin{addmargin}[1.5cm]{0cm}
    \begin{itemize}
        \item A interface de gerenciamento de cartões deve oferecer uma visão clara de todas as informações e operações disponíveis.
    \end{itemize}
\end{addmargin}
