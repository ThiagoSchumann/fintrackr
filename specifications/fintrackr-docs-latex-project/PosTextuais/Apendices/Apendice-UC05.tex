% ----------------------------------------------------------
\chapter{UC05 - Gerenciar Orçamentos}
% ----------------------------------------------------------
\label{apendiceUC05}

\subsection*{Identificador}
\textit{UC05}
% Conteúdo do Apêndice UC05

\subsection*{Título}
Gerenciar Orçamentos

\subsection*{Atores}
\begin{addmargin}[1.5cm]{0cm}
    \begin{itemize}
        \item Usuário
        \item Sistema de Gestão Financeira
    \end{itemize}
\end{addmargin}

\subsection*{Pré-condições}
\begin{addmargin}[1.5cm]{0cm}
    \begin{itemize}
        \item O usuário está autenticado e possui acesso à seção de orçamentos.
    \end{itemize}
\end{addmargin}

\subsection*{Fluxo Principal}
\begin{addmargin}[1.5cm]{0cm}
    \begin{enumerate}
        \item O usuário acessa a seção de orçamentos.
        \item O sistema exibe os orçamentos existentes categorizados por mês e categoria.
        \item O usuário seleciona um mês específico para visualizar ou editar.
        \item O sistema exibe detalhes do orçamento para o mês selecionado.
        \item O usuário pode adicionar, editar ou remover valores de orçamento para categorias específicas.
        \item O sistema valida e salva as alterações feitas pelo usuário.
    \end{enumerate}
\end{addmargin}

\subsection*{Fluxos Alternativos}
\begin{addmargin}[1.5cm]{0cm}
    \begin{itemize}
        \item \textbf{Criar Novo Orçamento}:
        \begin{enumerate}
            \item O usuário seleciona a opção para criar um novo orçamento.
            \item O sistema apresenta um formulário para inserir valores de orçamento por categoria.
            \item O usuário preenche os valores e submete.
            \item O sistema valida os dados e cria o novo orçamento.
        \end{enumerate}
    \end{itemize}
\end{addmargin}

\subsection*{Fluxos de Exceção}
\begin{addmargin}[1.5cm]{0cm}
    \begin{itemize}
        \item \textbf{Valores Negativos ou Inválidos}:
        \begin{enumerate}
            \item O sistema detecta valores negativos ou inválidos durante a edição ou criação de um orçamento.
            \item O sistema informa ao usuário sobre o erro e solicita correção.
        \end{enumerate}
    \end{itemize}
\end{addmargin}

\subsection*{Pós-condições}
\begin{addmargin}[1.5cm]{0cm}
    \begin{itemize}
        \item As alterações feitas pelo usuário nos orçamentos são salvas e refletidas no sistema.
    \end{itemize}
\end{addmargin}

\subsection*{Requisitos}
\begin{addmargin}[1.5cm]{0cm}
	\begin{itemize}
		\item \textbf{RF05}: Proporcionar a definição e monitoramento de orçamentos mensais por categoria, mostrando gastos realizados e disponíveis.
	\end{itemize}
\end{addmargin}

\subsection*{Regras de Negócio}
\begin{addmargin}[1.5cm]{0cm}
    \begin{itemize}
        \item \textbf{RN05}: Orçamentos não podem ter valores negativos.
        \item \textbf{RN12}: Usuários devem ser notificados ao atingir ou exceder orçamentos.
    \end{itemize}
\end{addmargin}

\subsection*{Notas e Comentários}
\begin{addmargin}[1.5cm]{0cm}
    \begin{itemize}
        \item A seção de orçamentos deve ser de fácil acesso e visualização, permitindo uma gestão rápida e eficiente dos valores.
    \end{itemize}
\end{addmargin}
